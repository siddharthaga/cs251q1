\documentclass{article}
\usepackage[english]{babel}
\usepackage{amsmath}

\usepackage{inputenc}
\usepackage{algorithm}
\usepackage{algpseudocode}
\usepackage{hyperref}
\usepackage{pdfpages}
\usepackage[backend=bibtex]{biblatex}
\bibliography{aut}
\title{\huge Procedure of multiplication of two-digit numbers.}

\date{6 march 2017}
\begin{document}
\begin{titlepage}
\maketitle
\end{titlepage}
\newpage
\section{Case 1} \label{case 1}
\noindent Case 1: The first digits are same, and the last ones add up to 10.
let two two-digit numbers be ``m'' and ``n''\footnote{these are exactly two-digit numbers}.``m'' has ``a'' at tens place and ``b'' at units.``n'' has ``c'' at tens place and ``b'' at units. ``a'' and ``c'' are same and c+d=10\cite{athar}

\noindent step1 : Simply mutiply ``a'' with its successive number \\
step2 : multiply last digit of both the numbers \\
step3 : write the numbers obtain in step1 and step2 together to form a new number\\

\noindent Below is the equation and after that its pseudocode


\begin{align*}
	z=a*(a+1)  \\
	y=b*d \\
	ans=zy
	\end{align*}
\subsection{pseudocode}
  \begin{algorithm}
   \caption{multiplication of two-digit numbers: }
    \begin{algorithmic}[1]
      \Function{Multiplication}{$m,n$}
	\State 

	\State Let $z$ , $y$, $a$ and $b$ be an integer
\State $a=m/10$
\State $b=m\%10$
\State $c=n/10$
\State $d=n\%10$
\If {$a == c$}
	\State $z=a*(a+1)$
	
	\State $y=b*d$
	\State $ans=(z*100)+y$ 
	
	\EndIf
	\EndFunction
\end{algorithmic}
\end{algorithm}
\noindent below is an example of \ref{case 1}:
\newline
\subsection{example}
\begin{tabular}{l} 
66\\

x64\\

6x(6+1):6x4 = 42:24 = 4224\\
\end{tabular}

\newpage
\section{Case 2}
\noindent Case  2: The first digits add up to 10, and the last ones are same.
let two two-digit numbers be ``m'' and ``n''\footnote{these are exactly two-digit numbers}.``m'' has ``a'' at tens place and ``b'' at units.``n'' has ``c'' at tens place and ``b'' at units. ``a'' and ``c'' are same and c+d=10\cite{Swami}

\noindent step1 : multiply first digit of both the number and add last digit to it \\
step2 : multiply last digit of both the numbers (which is same as taking square of last digit of any number)\\
step3 : write the numbers obtain in step1 and step2 together to form a new number\\

\noindent Below is the equation and after that its pseudocode

\begin{align*}
	z=a*c+d \\
	y=d*d \\
	ans=zy
	\end{align*}
\subsection{pseudocode}
  \begin{algorithm}
   \caption{multiplication of two-digit numbers: }
    \begin{algorithmic}[1]
      \Function{Multiplication}{$m,n$}
\State $a=m/10$
\State $b=m\%10$
\State $c=n/10$
\State $d=n\%10$
	\If {$b == d$}
	\State Let $z$ , $y$, $a$ and $b$ be an integer
	\State $z=a*c+d$
	
	\State$y=d*d$
	\State$ans=(z*100)+y$
	
	\EndIf
	\EndFunction
\end{algorithmic}
\end{algorithm}
\subsection{example}
\noindent below is an example\footnote{more examples can be found on \url{http://www.quickermaths.com/vedic-multiplication/}}:
\newline
\begin{tabular}{l} 

34\\

x74\\

3x7+4:4x4 = 25:16 = 2516\\
\end{tabular}
\newpage
\section{General case}
In case of three-digit numbers, multiply L.H.S. by 10 in  case 1. generalization of them is not so simple as it contain many other cases which can be found on \url{http://www.quickermaths.com/vedic-multiplication/}


%\begin{thebibliography}{9}
%\bibitem{swawmi} 
%Swami Bharati Krishna Tirtha 
%\textit{Vedic Mathematics}. 

%\end{thebibliography}
 %\bibliographystyle{acm}
\printbibliography
%\bibliography{aut}
\newpage
\includepdf[lastpage=1]{a7img.pdf}


\end{document}

